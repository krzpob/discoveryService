\documentclass[epic,eepic,aspectratio=169,12pt]{beamer}
\usepackage[polish]{babel}
\usepackage[T1]{polski}
\usepackage[utf8]{inputenc}
\usepackage[T1]{fontenc}
\usepackage{color}
\usepackage{picture}
\usepackage{graphicx}
\usepackage{minibox}
\usepackage{csquotes}
%\usetheme{Copenhagen}
\usecolortheme{crane}

\usepackage[]{csquotes}
\DeclareQuoteAlias{german}{polish}
\usepackage[%style=numeric %,authoryear,  alphabetic, authoryear, ect.
sorting=nty,
isbn=true,
backend=biber]{biblatex}

\title{Discovery Service}

\setbeamertemplate{bibliography item}{\insertbiblabel}

\author{Krzysztof Pobożan}

\begin{document}
	\begin{frame}
		\maketitle
	\end{frame}
	\begin{frame}{Agenda}
		\tableofcontents
	\end{frame}
	\section{Wstęp}
	\begin{frame}{Gdzie uruchamiamy?}
		Czy głównym problemem infrastruktury jest elastyczność i odporność na awarię sieci.
		Jeśli uruchamiamy w chmurze jak AWS czy Azure to pewne awarie w małej skali są nieuniknione.
	\end{frame}
	\begin{frame}{CAP}
		\begin{description}
			\item[Consistent] spójność danych
			\item[Partition] rozproszenie/rozdrobnienie
			\item[Aviability] dostępność
		\end{description}
		Systemy bazodanowe rozproszone nie są wstanie spełnić wszystkich elementów CAP na raz.
	\end{frame}
	\section{Zookeeper Discovery Service}
	\begin{frame}{Zookeeper Discovery Service}
		Zookeeper jest złym rozwiązaniem dla Discovery Service uruchamianym na chmurze.
		Przekłada spójność danych nad dostępność.
		Dla koordynacji to jest dobre, jednak dla discovery service, lepiej mieć informacje częściowo fałszywe niż nie mieć żadnych informacji.
		
	\end{frame}
\end{document}